                \chapter{基本理论方法}
\label{cha:sysu-thesis-contents-format-requirement}

\section{本征正交分解方法原理}

给定 $n$ 维空间中的 $m$ 个快照向量 $\mathbf{s}_i \in \mathbb{R}^n$ 构成数据矩阵 $S = [\mathbf{s}_1\ \mathbf{s}_2\ \cdots\ \mathbf{s}_m]$,POD 方法通过求解以下优化问题构建最优正交基:

\begin{equation}
    \max_{\{\phi_j\}} \sum_{j=1}^k \mathbb{E}[(\mathbf{s},\phi_j)^2] \quad \text{s.t.} \quad \phi_i^T\phi_j = \delta_{ij}
    \label{eq:pod_optimization}
\end{equation}

该优化问题的物理意义在于寻找一组正交基,使得这些基向量能够捕捉数据中最大的方差。换句话说,POD 方法试图用尽可能少的基向量来描述数据的主要特征,从而实现降维。

引入拉格朗日乘子 $\lambda_j$ 处理正交约束:

\begin{equation}
    \mathcal{L} = \sum_{j=1}^k \phi_j^T S S^T \phi_j - \sum_{j=1}^k \lambda_j (\phi_j^T\phi_j - 1)
    \label{eq:lagrangian}
\end{equation}

对 $\phi_j$ 求导并令导数为零,得到特征方程:

\begin{equation}
    \frac{\partial \mathcal{L}}{\partial \phi_j} = 2S S^T \phi_j - 2\lambda_j \phi_j = 0 \Rightarrow S S^T \phi_j = \lambda_j \phi_j
    \label{eq:eigen_equation}
\end{equation}

这表明 POD 基向量实际上是数据协方差矩阵的特征向量,对应的特征值衡量了每个基向量所包含的信息量。

\subsection{经典 POD 方法}
经典 POD (Classical POD) 是由 Lumley 在 1967 年提出的,它是最直接的 POD 实现方式。
\begin{equation}
R \phi_j = \lambda_j \phi_j, \quad \phi_j \in \mathbb{R}^n, \lambda_1 \geq \lambda_2 \geq \cdots \geq \lambda_n \geq 0
\end{equation}

其中,$R$ 代表协方差矩阵:

\begin{equation}
R = \frac{1}{m} \sum_{i=1}^m s_i s_i^T = S S^T \in \mathbb{R}^{n \times n}
\end{equation}

公式 (2.4) 所求得的特征向量 $\phi_j$ 就是 POD 基,它们是单位正交的,这也意味着:
\begin{equation}
\phi_j^T \phi_k = 
\begin{cases} 
0, & j \neq k \\
1, & j = k 
\end{cases}
\end{equation}

特征值 $\lambda_j$ 越大,就说明其对应的 POD 基 $\phi_j$ 包含这个矩阵 $S$ 的信息越多。将 POD 基按照其对应的特征值大小排序后,就可以得到有序的一组单位正交 POD 基。通常情况下,往往只保留前 k 个基来表达原始数据,在 $k$ 的选择上通常使用能量阈值法,设阈值为 $\epsilon_\sigma$,当满足式 (2.7) 时,确定 k 值,$\epsilon_\sigma$ 由设计者决定:

\begin{equation}
\frac{\sum_{j=1}^k \lambda_j}{\sum_{j=1}^n \lambda_j} \geq \epsilon_\sigma
\end{equation}

确定使用数量 k 后,用 POD 基来表达数据向量空间中的任意向量可得:

\begin{equation}
s \approx \sum_{j=1}^k \beta_j \phi_j
\end{equation}

其中,系数 $\beta_j$ 可以通过式 (2.9) 求出:

\begin{equation}
\beta_j = s^T \phi_j
\end{equation}
\subsection{快照 POD }
快照 POD 方法 (Snapshot POD) 是 Sirovich 等人提出的,也是目前气体设计领域最常用的 POD 方法。经典 POD 求解的实际上是 $R = S S^T$ 的特征值分解问题,$R$ 的大小是 $n \times n$。当 $n$ 的数值较大时,特征值分解所需消耗的时间将会显著增加。因此,当出现 $m \ll n$ 的时候就可以采用快照 POD 方法,快照 POD 方法求解的问题是:
\begin{equation}
S^T S \psi_j = \sigma_j \psi_j, \quad \psi_j \in \mathbb{R}^m, \sigma_1 \geq \sigma_2 \geq \cdots \geq \sigma_m \geq 0
\end{equation}

其中,$S^T S$ 是一个大小为 $m \times m$ 的协方差矩阵。在进行特征值分解时,$S^T S$ 和 $S S^T$ 的特征值都是相同的。另外,式 (2.10) 是从数学上对快照 POD 的描述,因此省去了 $\frac{1}{m}$ 的系数,

求出来的特征向量 $\psi_j$ 可以通过式 (2.11) 转换为 POD 基 $\phi_j$:

\begin{equation}\phi_{j}=S\psi_{j}\frac{1}{\sqrt{\lambda_{j}}}\in\mathbb{R}^{n},\quad j=1,2,\cdots,m\end{equation}

用矩阵形式描述为:

\begin{equation}
\Phi = S \Psi \Sigma^{-1/2}
\end{equation}

其中,$\Phi = [\phi_1 \phi_2 \cdots \phi_m] \in \mathbb{R}^{n \times m}$,$\Psi = [\psi_1 \psi_2 \cdots \psi_m] \in \mathbb{R}^{m \times m}$。
\subsection{SVD-POD}
对于数据矩阵 $S \in \mathbb{R}^{n \times m}$,实际上可以直接进行 SVD 分解:

\begin{equation}
S = \Phi \Sigma \Psi^T
\end{equation}

其中,$U$ 的列向量称为左奇异向量,$\Sigma$ 是奇异值矩阵,$V$ 的列向量称为右奇异向量。经济型 SVD 分解仅保留与数据秩相关的部分,避免计算冗余的零奇异值。

左奇异向量矩阵 $U$ 即为 POD 基:

\begin{equation}
    \sigma_j^2 = m\lambda_j, \quad \phi_j = U(:,j)
    \label{eq:svd_relation}
\end{equation}

这表明 SVD 分解中的左奇异向量与 POD 基在数学上是等价的,而奇异值的平方与 POD 特征值成正比。

通过矩阵分解理论可证:

\begin{equation}
    \mathbf{R} = \frac{1}{m}U\Sigma^2 U^T \Rightarrow U \equiv \Phi
    \label{eq:equivalence_proof}
\end{equation}

这表明经典 POD 方法中的协方差矩阵分解与 SVD 分解中的左奇异向量矩阵是相同的,从而证明了两种方法的等价性。


\section{矩阵特征值求解方法}

在科学技术的应用领域中,许多问题都可以归结为求解一个特征系统。如动力学系统和机构振动中的振动问题,求系统的频率与振型等等。矩阵特征值问题的数学表述如下:

设 \(\mathbf{A}\) 为 \(n\) 阶实矩阵,若有 \(n\) 维非零向量 \(\mathbf{x}\),有数 \(\lambda\) 使\begin{equation}
\mathbf{A}\mathbf{x} = \lambda \mathbf{x} 
\end{equation}

则称 \(\lambda\) 为 \(\mathbf{A}\) 的特征值,\(\mathbf{x}\) 为相应于 \(\lambda\) 的特征向量。线性代数理论中式通过求解特征多项式 \(\det(\mathbf{A} - \lambda \mathbf{I}) = 0\) 求得特征值,进而求得特征向量。当矩阵的阶数较高时,这种方法极为困难,所以常用数值方法来求解特征问题。常用的数值方法可以分为迭代法和变换法两种。变换法适用于中小型矩阵,常用的有适用于实对称矩阵的 Jacobi 方法,奇异值分解方法等等。

POD 方法中自相关矩阵均是实对称矩阵,而且一般阶数都不会很高,因此选用变换法中的 Jacobi 方法和奇异值分解求解矩阵的特征值与特征向量。在此对这两种方法做简单介绍。

\subsection{Jacobi 变换方法}

Jacobi 变换方法是求实对称矩阵全部特征值及对应特征向量的一种变换方法。其基本思想是把实对称矩阵经一系列正交相似变换约化为一个近似对角阵,即:

\begin{equation}
\left\{
\begin{aligned}
\mathbf{A}_0 &= \mathbf{A} \\
\mathbf{A}_{k+1} &= \mathbf{R}_{k+1}^T \mathbf{A}_k \mathbf{R}_{k+1} \quad k = 0, 1, 2, \ldots
\end{aligned}
\right. 
\end{equation}

最终对角阵的对角元就是该矩阵的近似特征值,由各个正交变换阵的乘积可得对应的特征向量。

一般选取 Givens 旋转矩阵做相似变换阵。该矩阵定义如下:


\begin{equation}
\mathbf{R}(p,q,\theta)=\begin{bmatrix} 1 & & & & & & & \\ & \ddots & & & & & & \\ & & 1 & & & & & \\ & & & \cos\theta & \cdots & \sin\theta & & \\ & & & \vdots & \vdots & \vdots & & \\ & & & -\sin\theta & \cdots & \cos\theta & & \\ & & & & & & \ddots & \\ & & & & & & & 1 \\\end{bmatrix} 
\end{equation}
它是在单位阵 \(\mathbf{I}\) 的 \(p\) 行、\(q\) 行和 \(p\) 列、\(q\) 列的四个交叉位置上分别置 \(\cos\theta\),\(\sin\theta\),\(-\sin\theta\) 和 \(\cos\theta\) 而成的。容易验证旋转矩阵是正交矩阵,即 \(\mathbf{R}^T(p, q, \theta) = \mathbf{R}^{-1}(p, q, \theta)\),所以用它做相似变换阵时十分方便。

综上,Jacobi 变换法的计算步骤为:

\begin{enumerate}
    \item 令 \(\mathbf{H}_{k+1}^T = \mathbf{I}\),\(k = 0\),确定 \(\varepsilon\);
    \item 确定 \(p\),\(q\)(\(p < q\)),使 \(|a_{pq}^{(k)}| = \max_{i < j} |a_{ij}^{(k)}|\);
    \item 若 \(|a_{pq}^{(k)}| < \varepsilon\),则停止计算,对角矩阵的元素即为特征值,\(\mathbf{H}_k\) 的列向量即为特征向量;
    \item 计算 \(\cos\theta\),\(\sin\theta\),获得 Givens 变换阵 \(\mathbf{R}_{pq}\):
    
    通常将 \(\theta\) 限制在 \(\left[ -\frac{\pi}{4}, \frac{\pi}{4} \right]\),具体求法如下:
    
    当 \(a_{pp}^{(k)} = a_{qq}^{(k)}\),取 \(\cos\theta = \frac{\sqrt{2}}{2}\),\(\sin\theta = \text{sign}(a_{pq}^{(k)}) \cos\theta\);
    
    当 \(a_{pp}^{(k)} \neq a_{qq}^{(k)}\),取 \(\cos\theta = \frac{1}{\sqrt{1 + t^2}}\),\(\sin\theta = t \cos\theta\),其中 \(t = \tan\theta = \frac{\text{sign}(d)}{|d| + \sqrt{d^2 + 1}}\),\(d = \frac{a_{pp}^{(k)} - a_{qq}^{(k)}}{2 a_{pq}^{(k)}}\)。\(a\) 是迭代中所求矩阵的元素,\(k\) 表示迭代的步数。
    \item 计算 \(\mathbf{H}_{k+1}^T = \mathbf{H}_k^T \mathbf{R}_{pq}\),\(\mathbf{A}_{k+1} = \mathbf{R}_{pq}^T \mathbf{A}_k \mathbf{R}_{pq}\),返回(2)。
\end{enumerate}

\subsection{奇异值分解方法}

对于实矩阵 \(\mathbf{A} \in \mathbb{R}^{m \times n}\),\(\mathbf{A}^T \mathbf{A}\) 的特征值为 \(\lambda_1 \geq \lambda_2 \geq \cdots \geq \lambda_n \geq 0\),则称 \(\sigma_i = \sqrt{\lambda_i}\)(\(i = 1, 2, \ldots, n\))为 \(\mathbf{A}\) 的奇异值。对于上述矩阵 \(\mathbf{A}\),存在 \(m\) 阶正交矩阵 \(\mathbf{U}\) 和 \(n\) 阶正交矩阵 \(\mathbf{V}\),使得:

\begin{equation}
\mathbf{U}^T \mathbf{A} \mathbf{V} = \begin{pmatrix} \mathbf{\Sigma} & \mathbf{0} \\ \mathbf{0} & \mathbf{0} \end{pmatrix} 
\end{equation}

其中 \(\mathbf{\Sigma} = \text{diag}(\sigma_1, \sigma_2, \ldots, \sigma_r)\),将上式改写为:

\begin{equation}
\mathbf{A} = \mathbf{V} \begin{pmatrix} \mathbf{\Sigma} & \mathbf{0} \\ \mathbf{0} & \mathbf{0} \end{pmatrix} \mathbf{U}^T 
\end{equation}

称之为 \(\mathbf{A}\) 的奇异值分解。

设 \(\mathbf{A} \in \mathbb{R}^{n \times n}\) 为实对称矩阵,其特征值为 \(\lambda_i(\mathbf{A})\)(\(i = 1, 2, \ldots, n\)),因为对实对称矩阵有 \(\mathbf{A} = \mathbf{A}^T\),所以:

\begin{equation}
\lambda_i(\mathbf{A}^T \mathbf{A}) = \lambda_i(\mathbf{A}^2) = \left[ \lambda_i(\mathbf{A}) \right]^2 \quad (i = 1, 2, \ldots, n) 
\end{equation}
A 的奇异值 \(\sigma_i = \sqrt{\lambda_i(\mathbf{A}^T \mathbf{A})} = \lambda_i(\mathbf{A})\)(\(i = 1, 2, \ldots, n\))。至此,我们可以知道,对于实对称矩阵,它的奇异值和它的特征值是相同的。只要得到 A 的奇异值分解:

\begin{equation}
\mathbf{A} = \mathbf{V}^T \mathbf{\Sigma} \mathbf{U} 
\end{equation}

则 A 的特征值为 \(\mathbf{\Sigma}\) 中的对角元素,对应特征向量为 \(\mathbf{V}^T\) 的列向量。

一般实矩阵的奇异值分解要首先使用 Householder 变换把矩阵化为双对角线矩阵,再用 QR 算法迭代求解奇异值,上述两种方法较为常见,本文不再赘述。
\section{气动分析方法}
使用 POD 方法的前提是有准确的流场求解方法,以提供合理的采样解。本文所作的工作都是针对要型流场进行的,其中进行方法验证时一般都使用的欧拉方程求解器,以缩短验算周期。在与优化相关的问题中使用的是 NS 方程求解器。鉴于二维定常 NS 方程的数值求解技术已臻成熟,且本文重点不在于此,故只对二维定常 NS 方程的数值求解方法做简单介绍。

从欧拉运动描述的观点出发,在任何有限控制体内的流动,质量、动量、能量所发生的暂时性变化,必须通过流入流出控制体的通量和控制体内部的当地变化来平衡。对于大多空气动力学中的应用来说,流动内部不存在热源或质量源,通常情况下,体积力(或彻体力)的影响,例如重力是可以忽略不计的。于是,控制方程式可以写成如下积分形式:

\begin{equation}
\frac{\partial}{\partial t} \iiint_V \mathbf{W} \, dV + \iint_S \mathbf{F} \cdot \mathbf{n} \, dS = 0 
\end{equation}

式中,\(V\) 是一个有封闭边界 \(S\) 的任意控制体,\(\mathbf{n}\) 为边界的法向量;指向朝外,状态变量 \(\mathbf{W}\) 的表达式为:

\begin{equation}
\mathbf{W} = (\rho, \rho u, \rho v, \rho E)^T 
\end{equation}

其中 \(\rho\) 为密度,\(u, v\) 是两个个笛卡尔速度分量,\(E\) 是单位质量流体的总能量,表达式为:

\begin{equation}
E = e + \frac{1}{2}(u^2 + v^2) 
\end{equation}

其中 \(e\) 是内能,方程(4-1)中的矢通量 \(\mathbf{F}\) 可分解成对流项(无粘部分)\(\mathbf{F}_c\) 和耗散项(粘性和热传递部分)\(\mathbf{F}_d\),即:
\begin{equation}
\mathbf{F} = \mathbf{F}_c + \mathbf{F}_d 
\end{equation}

对流通量表达式为:

\begin{equation}
\mathbf{F}_c = 
\begin{bmatrix}
\rho u i & \rho v j \\
(\rho u^2 + p) i & \rho u v j \\
\rho v u i & (\rho v^2 + p) j \\
(\rho E u + \rho u) i & (\rho E v + \rho v) j
\end{bmatrix} 
\end{equation}

其中,\(i, j\) 是直角坐标系的单位向量,通量的粘性应力项和热扩散的状态方程为:

\begin{equation}
\mathbf{F}_d = 
\begin{bmatrix}
0 & 0 \\
\tau_{x,i} & \tau_{x,j} \\
\tau_{y,i} & \tau_{y,j} \\
(u \tau_x + v \tau_y - q_x) i & (u \tau_x + v \tau_y - q_y) j
\end{bmatrix} 
\end{equation}

各个方向的剪应力及热通量可被定义为:
\begin{align}
\tau_{x,i} &= 2(\mu + \mu_t) \frac{\partial u}{\partial x} - \frac{2}{3} (\mu + \mu_t) \left( \frac{\partial u}{\partial x} + \frac{\partial v}{\partial y} \right) \\
\tau_{y,i} &= 2(\mu + \mu_t) \frac{\partial v}{\partial y} - \frac{2}{3} (\mu + \mu_t) \left( \frac{\partial u}{\partial x} + \frac{\partial v}{\partial y} \right) \\
\tau_{x,j} &= \tau_{x,i} = (\mu + \mu_t) \left( \frac{\partial u}{\partial y} + \frac{\partial v}{\partial x} \right)  \\
q_x &= -(k + k_t) \frac{\partial T}{\partial x} \\
q_y &= -(k + k_t) \frac{\partial T}{\partial y}
\end{align}

其中 \(\mu\) 表示层流粘性系数,\(\mu_t\) 表示湍流粘性系数,\(k\) 表示层流导热系数,\(k_t\) 表示湍流导热系数,\(T\) 表示温度。

总焓可以表示为:

\begin{equation}
H = h + \frac{1}{2}(u^2 + v^2)
\end{equation}

完全气体的定义和关系由下面的状态方程给出:

\begin{equation}
e = c_v T, \quad h = c_p T, \quad R = c_p - c_v, \quad \gamma = \frac{c_p}{c_v} 
\end{equation}

理想气体状态方程如下:

\begin{equation}
\frac{p}{\rho} = RT 
\end{equation}
压力和总焓可被表示为:

\begin{equation}
H = E + \frac{p}{\rho} 
\end{equation}

即
\begin{equation}
p = (\gamma - 1) \rho \left[ E - \frac{1}{2}(u^2 + v^2) \right] 
\end{equation}

其中,\(c_p\) 和 \(c_v\) 分别表示恒压和恒容比热,\(\gamma\) 是比热比,\(R\) 是气体常数,温度 \(T\) 由此给出:
\begin{equation}
T = \frac{\gamma}{c_v (\gamma - 1) \rho} p 
\end{equation}

由 Sutherland's 方程得到层流粘性系数公式:

\begin{equation}
\frac{\mu}{\mu_\infty} = \left( \frac{T}{T_\infty} \right)^{\frac{3}{2}} \frac{T_\infty + 110.3}{T + 110.3} 
\end{equation}

最终,完全气体声速为:

\begin{equation}
a = \sqrt{\frac{\gamma p}{\rho}} 
\end{equation}

马赫数为:

\begin{equation}
M = \frac{\sqrt{u^2 + v^2 + w^2}}{a}
\end{equation}

该方程在数值求解时:空间离散使用格心格式的有限体积法,时间推进采用显式 5 步龙格库塔迭代方法。使用 \(k-\omega\) 端流模型。物面边界使用绝热的无滑移边界条件,远场边界使用无反射边界条件。加入人工粘性项以保证算法的稳定及抑制激波附近的震荡。使用隐式残值光顺和多重网格的加速收敛技术。

\section{单变量三次样条插值法}
\subsection{数学描述}
考虑定义在区间$[a,b]$上的三次样条插值问题。给定节点$\{x_0,x_1,...,x_n\}$及其对应函数值$\{f(x_i)\}$,要求构造分段三次多项式函数$S(x)$满足:

\begin{enumerate}
    \item 在子区间$[x_i, x_{i+1}]$上的局部表达式为:
    \begin{equation}
        S_i(x) = a_i + b_i(x-x_i) + c_i(x-x_i)^2 + d_i(x-x_i)^3
    \end{equation}
    \item 整体满足$C^2$连续性:
    \begin{align}
        \lim_{x\to x_i^+} S^{(k)}(x) &= \lim_{x\to x_i^-} S^{(k)}(x), \quad k=0,1,2
    \end{align}
    \item 自然边界条件:
    \begin{equation}
        S''(x_0) = S''(x_n) = 0
    \end{equation}
\end{enumerate}
\subsection{系数求解}
令$M_i = S''(x_i)$,根据三次多项式的二阶导数特性,可得递推关系:
\begin{equation}
    \frac{h_{i-1}}{6}M_{i-1} + \frac{h_{i-1}+h_i}{3}M_i + \frac{h_i}{6}M_{i+1} = \frac{f(x_{i+1})-f(x_i)}{h_i} - \frac{f(x_i)-f(x_{i-1})}{h_{i-1}}
\end{equation}
其中$h_i = x_{i+1}-x_i$。该方程组可表示为三对角矩阵形式:
\begin{equation}
    \begin{pmatrix}
        2 & \lambda_1 & 0 & \cdots & 0 \\
        \mu_2 & 2 & \lambda_2 & \cdots & 0 \\
        \vdots & \ddots & \ddots & \ddots & \vdots \\
        0 & \cdots & \mu_{n-1} & 2 & \lambda_{n-1} \\
        0 & \cdots & 0 & \mu_n & 2
    \end{pmatrix}
    \begin{pmatrix}
        M_1 \\ M_2 \\ \vdots \\ M_{n-1} \\ M_n
    \end{pmatrix}
    = \mathbf{b}
\end{equation}
其中$\mu_i = h_{i-1}/(h_{i-1}+h_i)$,$\lambda_i = 1-\mu_i$,右端项$\mathbf{b}$由差商计算。通过Thomas算法可高效求解该方程组。

对于形如$A\mathbf{M} = \mathbf{b}$的三对角方程组:
\[
\begin{pmatrix}
    b_1 & c_1 & 0 & \cdots & 0 \\
    a_2 & b_2 & c_2 & \cdots & 0 \\
    0 & \ddots & \ddots & \ddots & \vdots \\
    \vdots & & a_{n-1} & b_{n-1} & c_{n-1} \\
    0 & \cdots & 0 & a_n & b_n
\end{pmatrix}
\begin{pmatrix}
    M_1 \\ M_2 \\ \vdots \\ M_{n-1} \\ M_n
\end{pmatrix}
= 
\begin{pmatrix}
    d_1 \\ d_2 \\ \vdots \\ d_{n-1} \\ d_n
\end{pmatrix}
\]
采用追赶法(Thomas算法)进行高效求解,其计算步骤如下:

\begin{enumerate}
    \item {向前消元}:计算修正系数
    \begin{align}
        c'_1 &= \frac{c_1}{b_1} \\
        d'_1 &= \frac{d_1}{b_1} \\
        c'_i &= \frac{c_i}{b_i - a_i c'_{i-1}}, \quad i=2,...,n-1 \\
        d'_i &= \frac{d_i - a_i d'_{i-1}}{b_i - a_i c'_{i-1}}, \quad i=2,...,n
    \end{align}

    \item{向后回代}:求解节点二阶导数
    \begin{align}
        M_n &= d'_n \\
        M_i &= d'_i - c'_i M_{i+1}, \quad i=n-1,n-2,...,1
    \end{align}
\end{enumerate}

在三次样条插值的具体场景中:
\begin{itemize}
    \item 主对角线元素:$b_i = \frac{h_{i-1} + h_i}{3}$
    \item 次对角线元素:$a_i = \frac{h_{i-1}}{6}$
    \item 超对角线元素:$c_i = \frac{h_i}{6}$
    \item 右端项:$d_i = \frac{f(x_{i+1}) - f(x_i)}{h_i} - \frac{f(x_i) - f(x_{i-1})}{h_{i-1}}$
\end{itemize}

该算法的时间复杂度为$O(n)$,显著优于高斯消元法的$O(n^3)$,特别适用于大规模网格计算\cite{deboor1978}。
\section{双变量三次样条插值法}
\subsection{基本定义与条件}
设矩形区域$U = [a,b] \times [c,d]$被划分为$m \times n$个网格,双变量三次样条插值函数$S(x,y)$需满足:
\begin{enumerate}
    \item {局部双三次性}:在子区域$[x_i,x_{i+1}] \times [y_j,y_{j+1}]$内,$S(x,y)$为双三次多项式
    \item{全局光滑性}:$S(x,y) \in C^{2,2}(U)$,即二阶混合偏导数连续
    \item {插值条件}:$S(x_i,y_j) = z_{ij},\ \forall i=0,1,...,m; j=0,1,...,n$
\end{enumerate}
\subsection{张量积构造方法}
基于一维三次样条的张量积扩展,构造方法可分两步实现:

\begin{enumerate}
    \item 固定$y = y_j$,构造$x$方向三次样条:
    \begin{equation}
        S_j(x) = \sum_{k=0}^3 a_{jk}(x-x_i)^k,\quad x \in [x_i,x_{i+1}]
    \end{equation}
    系数$a_{jk}$通过以下条件确定:
    \begin{align}
        S_j(x_i) &= z_{ij} \\
        S_j'(x_i^+) &= S_j'(x_i^-) \\
        S_j''(x_i^+) &= S_j''(x_i^-)
    \end{align}
    
    \item 对每个$x_i$构造$y$方向样条:
    \begin{equation}
        S_i(y) = \sum_{l=0}^3 b_{il}(y-y_j)^l,\quad y \in [y_j,y_{j+1}]
    \end{equation}
    最终的双变量样条表示为:
    \begin{equation}
        S(x,y) = \sum_{i=1}^m \sum_{j=1}^n \alpha_{ij} B_i^{(3)}(x) B_j^{(3)}(y)
    \end{equation}
    其中$B^{(3)}$为三次B样条基函数\cite{deboor1978}。
\end{enumerate}
\subsection{连续性条件}
为保证$C^{2,2}$连续性,需满足以下条件:
\begin{enumerate}
    \item 函数值连续:$\lim_{\epsilon \to 0} S(x_i \pm \epsilon, y_j) = z_{ij}$
    \item 一阶偏导连续:
    \begin{equation}
        \frac{\partial S}{\partial x}\bigg|_{(x_i^+,y_j)} = \frac{\partial S}{\partial x}\bigg|_{(x_i^-,y_j)}
    \end{equation}
    \item 二阶混合偏导连续:
    \begin{equation}
        \frac{\partial^2 S}{\partial x \partial y}\bigg|_{(x_i^+,y_j^+)} = \frac{\partial^2 S}{\partial x \partial y}\bigg|_{(x_i^-,y_j^-)}
    \end{equation}
\end{enumerate}
\subsection{算法实现}
核心算法流程如下:
\begin{enumerate}
    \item 输入:网格节点$\{(x_i,y_j)\}_{i,j=0}^{m,n}$,对应函数值$\{z_{ij}\}$
    \item 对每个$y_j$,求解三对角方程组:
    \begin{equation}
        \begin{bmatrix}
        2 & 1 & & \\
        1 & 4 & 1 & \\
        & \ddots & \ddots & \ddots \\
        & & 1 & 2
        \end{bmatrix}
        \begin{bmatrix}
        M_{1j} \\ M_{2j} \\ \vdots \\ M_{mj}
        \end{bmatrix}
        = 6
        \begin{bmatrix}
        \frac{z_{1j}-z_{0j}}{h_1} - \frac{z_{0j}-z_{-1j}}{h_0} \\
        \vdots \\
        \frac{z_{mj}-z_{m-1j}}{h_m} - \frac{z_{m-1j}-z_{m-2j}}{h_{m-1}}
        \end{bmatrix}
    \end{equation}
    其中$M_{ij}$为二阶导数在节点处的值
    \item 类似地构造$y$方向样条
    \item 合成双变量样条表达式
\end{enumerate}


% \subsection{正文}

% 正文是毕业论文的主体和核心部分,不同学科专业和不同的选题可以有不同的写作方式。正文一般包括以下几个方面:

% \begin{enumerate}
%     \item \textbf{绪论} \\
%           绪论应包括毕业论文选题的背景、目的和意义;对国内外研究现状和相关领域中已有的研究成果的简要评述;介绍本项研究工作研究设想、研究方法或实验设计、理论依据或实验基础;涉及范围和预期结果等。要求言简意赅,注意不要与摘要雷同或成为摘要的注解。
%     \item \textbf{主体} \\
%           论文主体是毕业论文的主要部分,必须言之成理,论据可靠,严格遵循本学科国际通行的学术规范。在写作上要注意结构合理、层次分明、重点突出,章节标题、公式图表符号必须规范统一。论文主体的内容根据不同学科有不同的特点,一般应包括以下几个方面:
%           \begin{enumerate}
%               \item 毕业论文总体方案或选题的论证;
%               \item 毕业论文各部分的设计实现,包括实验数据的获取、数据可行性及有效性的处理与分析、各部分的设计计算等;
%               \item 对研究内容及成果的客观阐述,包括理论依据、创新见解、创造性成果及其改进与实际应用价值等;
%               \item 论文主体的所有数据必须真实可靠,凡引用他人观点、方案、资料、数据等,无论曾否发表,无论来源于纸质或电子版材料,均应详加注释。自然科学论文应推理正确、结论清晰;人文和社会学科的论文应把握论点正确、论证充分、论据可靠,恰当运用系统分析和比较研究的方法进行模型或方案设计,注重实证研究和案例分析,根据分析结果提出建议和改进措施等。
%           \end{enumerate}
%     \item \textbf{结论} \\
%           结论是毕业论文的总结,是整篇论文的归宿,应精炼、准确、完整。结论应着重阐述自己的创造性成果及其在本研究领域中的意义和作用,还可进一步提出需要讨论的问题和建议。
% \end{enumerate}

% \subsection{参考文献}

% 参考文献是毕业论文不可缺少的组成部分,它反映毕业论文的取材来源、材料的广博和可靠程度,也是作者对他人知识成果的承认和尊重。凡有引用他人的著作、论文等,均应列于参考文献中。

% \subsection{相关的科研成果目录}

% 本科期间发表的与毕业论文相关的论文或被鉴定的技术成果、发明专利等,应在成果目录中列出。此项不是必需项,空缺时可以省略。

% \subsection{附录}

% 对于一些不宜放在正文中的重要支撑材料,可编入毕业论文的附录中,包括某些重要的原始数据、详细数学推导、程序全文及其说明、复杂的图表、设计图纸等一系列需要补充提供的说明材料。如果毕业论文中引用的实例、数据资料,实验结果等符号较多时,为了节约篇幅,便于读者查阅,可以编写一个符号说明,注明符号代表的意义。附录的篇幅不宜太多,一般不超过正文。此项不是必需项,空缺时可以省略。

% \subsection{致谢}

% 致谢应以简短的文字对课题研究与论文撰写过程中曾直接给予帮助的人员(例如指导教师、答疑教师及其他人员)表达自己的谢意,这不仅是一种礼貌,也是对他人劳动的尊重,是治学者应当遵循的学术规范。内容限一页。


% \section{毕业论文的撰写格式要求}


% \subsection{文字和字数}


% 除外国语言文学类专业外,其他专业的毕业论文须采用简化汉语文字撰写。论文正文部分一般不少于8000 字,各专业可根据需要确定具体的字数要求,并报教务部备案。

% \subsection{字体和字号}

% 标题一般用黑体,内容一般用宋体,数字和英文字母一般用Times New Roman,具体如\autoref{tab:font-spec}。

% \begin{table}[]
%     \caption{字体使用规范}
%     \begin{tabular}{|c|c|}
%         \hline
%         论文题目               & 黑体二号居中                                  \\ \hline
%         中文摘要标题           & 黑体三号居中                                  \\ \hline
%         中文摘要内容           & 宋体小四号                                    \\ \hline
%         中文关键词             & 宋体小四号(标题``关键词''加粗)              \\ \hline
%         英文摘要标题           & Times New Roman加粗三号全部大写               \\ \hline
%         英文摘要内容           & Times New Roman小四号                         \\ \hline
%         英文关键词             & Times New Roman小四号(标题``Keywords''加粗) \\ \hline
%         目录标题               & 黑体三号居中                                  \\ \hline
%         目录内容               & 宋体小四号                                    \\ \hline
%         正文各章标题           & 黑体三号居中                                  \\ \hline
%         正文各节一级标题       & 黑体四号左对齐                                \\ \hline
%         正文各节二级及以下标题 & 宋体小四号加粗左对齐空两格                    \\ \hline
%         正文内容               & 宋体小四号                                    \\ \hline
%         参考文献标题           & 黑体三号居中                                  \\ \hline
%         参考文献内容           & 宋体五号                                      \\ \hline
%         致谢、附录标题         & 黑体三号居中                                  \\ \hline
%         致谢、附录内容         & 宋体小四号                                    \\ \hline
%         页眉与页脚             & 宋体五号居中                                  \\ \hline
%         图题、表题             & 宋体五号                                      \\ \hline
%         脚注、尾注             & 宋体小五号                                    \\ \hline
%     \end{tabular}
%     \label{tab:font-spec}
% \end{table}


% 字体样例可见如下(以居中形式展现): \\



% \begin{center}
%     {\heiti\zihao{2}黑体二号居中} \\

%     {\heiti\zihao{3}黑体三号居中} \\

%     {\heiti\zihao{4}黑体四号居中} \\

%     {\songti\zihao{4}宋体四号居中} \\

%     {\songti\zihao{-4}宋体小四号居中} \\

%     {\songti\zihao{5}宋体五号居中} \\

%     {\songti\zihao{-5}宋体小五号居中} \\

%     {\zihao{3} Times New Roman : Three} 三号居中 \\

%     {\zihao{-4} Times New Roman : Small Four} 小四号居中 \\

% \end{center}


% \subsection{页面设置}

% 纸张大小:A4。

% 页边距:上边距25 mm,下边距20 mm,左右边距均为30 mm。

% 行距:1.5倍行距,章和节标题段前段后各空0.5行。

% \subsection{页码}

% 页面底端居中,从摘要开始至绪论之前以大写罗马数字(
% \uppercase\expandafter{\romannumeral1} ,
% \uppercase\expandafter{\romannumeral2} ,
% \uppercase\expandafter{\romannumeral3} ,
% …)单独编连续码,绪论开始至论文结尾,以阿拉伯数字(1,2,3…)编连续码。

% \subsection{关键词}


% 摘要正文下方另起一行顶格打印``关键词''款项,后加冒号,多个关键词以逗号分隔。

% \subsection{目录}

% 目录应另起一页,包括论文中的各级标题,按照``一……''、``(一)……''或``1……''、``1.1……''格式编写。

% \subsection{各级标题}

% 正文各部分的标题应简明扼要,不使用标点符号。论文内文各大部分的标题用``一、二……(或1、2……)'',次级标题为``(一)、(二)……(或1.1、2.1……)'',三级标题用``1、2……(或1.1.1、2.1.1……)'',四级标题用``(1)、(2)……(或1.1.1.1、2.1.1.1……)'',不再使用五级以下标题。两类标题不要混编。

% \subsection{名词术语}

% \begin{enumerate}
%     \item 科学技术名词术语尽量采用全国自然科学名词审定委员会公布的规范词或国家标准、部标准中规定的名称,尚未统一规定或叫法有争议的名词术语,可采用惯用的名称。
%     \item 特定含义的名词术语或新名词、以及使用外文缩写代替某一名词术语时,首次出现时应在括号内注明其含义,如:经济合作与发展组织(Organisation for Economic Co-operation and Development, OECD)。
%     \item 外国人名一般采用英文原名,可不译成中文,英文人名按姓前名后的原则书写,如:CRAY P,不可将外国人姓名中的名部分漏写,例如:不能只写CRAY, 应写成CRAY P。一般很熟知的外国人名(如牛顿、爱因斯坦、达尔文、马克思等)可按通常标准译法写译名。
% \end{enumerate}

% \subsection{物理量名称、符号与计量单位}

% \begin{enumerate}
%     \item 论文中某一物理量的名称和符号应统一,应采用国务院发布的《中华人民共和国法定计量单位》、国际公认或各行业领域惯用的计量单位。单位名称和符号的书写方式,应采用国际通用符号。
%     \item 在不涉及具体数据表达时允许使用中文计量单位如``千克''。
%     \item 表达时刻应采用中文计量单位,如``下午3点10分'',不能写成``3h10min'',在表格中可以用``3:10PM''表示。
%     \item 物理量符号、物理量常量、变量符号用斜体,计量单位符号均用正体。
% \end{enumerate}

% \subsection{数字}

% \begin{enumerate}
%     \item 无特别约定情况下,一般均采用阿拉伯数字表示。
%     \item 年份一律使用4位数字表示。
%           % \item 统计符号的格式:一般除$\mu$、$\alpha$、$\beta$、$\lambda$、$\varepsilon$ 以及$V$等符号外,其余统计符号一律以斜体字呈现,如$ANCOVA$,$ANOVA$,$MANOVA$,$N$,$nl$,$M$,$SD$,$F$,$p$,$r$等。
%     \item 统计符号的格式:一般除μ、α、β、λ、ε以及V等符号外,其余统计符号一律以斜体字呈现,如\textit{ANCOVA},\textit{ANOVA},\textit{MANOVA},\textit{N},\textit{nl},\textit{M},\textit{SD},\textit{F},\textit{p},\textit{r}等。
% \end{enumerate}


% \subsection{公式}

% \begin{enumerate}
%     \item 公式应另起一行写在稿纸中央。一行写不完的长公式,最好在等号处转行,如做不到这一点,可在运算符号(如``+''、``-''号)处转行,等号或运算符号应在转行后的行首。
%     \item 公式的编号用圆括号括起,放在公式右边行末,在公式和编号之间不加虚线。公式可按全文统编序号,也可按章编独立序号,如(49)、(4.11)、(4-11)等。采用哪一种序号应和图序、表序编法一致。不应出现某章里的公式编序号,有的则不编序号。子公式可不编序号,需要引用时可加编a、b、c……,重复引用的公式不得另编新序号。公式序号必须连续,不得重复或跳缺。
%     \item 文中引用某一公式时,可写成``由式(序号)''。
% \end{enumerate}

% \ \\

% 这是一个例子:

% \begin{equation}
%     \label{eq:example-formulas}
%     ax^2 +bx+c = 0
% \end{equation}


% 如\autoref{eq:example-formulas}所示,为了求解该一元二次方程,我们可以推导得到该方程的求根公式。因此,由\autoref{eq:example-formulas2}即可求解该方程的两个根。

% \begin{equation}
%     \label{eq:example-formulas2}
%     \begin{split}
%         x_1 = & \frac{-b+\sqrt{b^2-4ac}}{2a} \\
%         & \\
%         x_2 = & \frac{-b-\sqrt{b^2-4ac}}{2a}
%     \end{split}
% \end{equation}

% \subsection{表格}

% \begin{enumerate}
%     \item 表格必须与论文叙述有直接联系,不得出现与论文叙述脱节的表格。表格中的内容在技术上不得与正文矛盾。
%     \item 每个表格都应有自己的标题和序号。标题应写在表格上方正中,不加标点,序号写在标题左方。
%     \item 全文的表格可以统一编序,也可以逐章单独编序。采用哪一种方式应和插图、公式的编序方式统一。表序必须连续,不得跳缺。
%     \item 表格允许下页接写,接写时标题省略,表头应重复书写,并在右上方写``续表××''。多项大表可以分割成块,多页书写,接口处必须注明``接下页''、``接上页''、``接第×页''字样。
%     \item 表格应放在离正文首次出现处最近的地方,不应超前和过分拖后。
% \end{enumerate}


% 例子可见\autoref{tab:table-example}。

% \begin{table}[!htbp]
%     \centering
%     \caption{表格例子}
%     \label{tab:table-example}
%     \begin{tabular}{|l|l|}
%         \hline
%         \multicolumn{1}{|c|}{这是表格第一行第一列} & 这是表格第一行第二列 \\ \hline
%         这是表格第二行第一列                       & 这是表格第二行第二列 \\ \hline
%     \end{tabular}
% \end{table}

% \subsection{图}


% \begin{enumerate}
%     \item 插图应与文字内容相符,技术内容正确。所有制图应符合国家标准和专业标准。对无规定符号的图形应采用该行业的常用画法。
%     \item 每幅插图应有标题和序号,全文的插图可以统一编序,也可以逐章单独编序,采取哪一种方式应和表格、公式的编序方式统一。图序必须连续,不重复,不跳缺。
%     \item 由若干分图组成的插图,分图用a、b、c……标序。分图的图名以及图中各种代号的意义,以图注形式写在图题下方,先写分图名,另起行写代号的意义。
%     \item 图与图标题、图序号为一个整体,不得拆开排版为两页。当页空白不够排版该图整体时,可将其后文字部分提前,将图移至次页最前面。
%     \item 对坐标轴必须进行文字标示,有数字标注的坐标图必须注明坐标单位。
% \end{enumerate}

% 例子可见\autoref{fig:example-figure}。

% \begin{figure}[!htbp]
%     \centering
%     \includegraphics[width=3cm]{example-image-a}
%     \caption{图片例子}
%     \label{fig:example-figure}
% \end{figure}

% \subsection{注释}

% 毕业论文(设计)中有个别名词或情况需要解释时,可加注说明。注释采用脚注或尾注,应根据注释的先后顺序编排序号。注释序号以``①、②''等数字形式标示在正文中被注释词条的右上角,脚注或尾注内容中的序号应与被注释词条序号保持一致。

% 脚注例子可见这里\footnote{这是一个脚注}。

% \subsection{参考文献}

% 参考文献的序号左顶格,并用数字加方括号表示,如``[1]''。每一条参考文献著录均以``.''结束。各类参考文献的具体编排格式请参照国家标准《信息与文献 参考文献著录规则》(GB/T 7714-2015)。

% 参考文献例子可见这里\cite{sysu-thesis}。


% \subsection{附录}

% 论文附录依次用大写字母``附录A、附录B、附录C……''表示,附录内的分级序号可采用``附A1、附A1.1、附A1.1.1''等表示,图、表、公式均依此类推为``图A1、表A1、式A1''等。

% % TODO:增加引用例子。