%%
% 摘要信息
% 本文档中前缀"c-"代表中文版字段, 前缀"e-"代表英文版字段
% 摘要内容应概括地反映出本论文的主要内容,主要说明本论文的研究目的、内容、方法、成果和结论。要突出本论文的创造性成果或新见解,不要与引言相 混淆。语言力求精练、准确,以 300—500 字为宜。
% 在摘要的下方另起一行,注明本文的关键词(3—5 个)。关键词是供检索用的主题词条,应采用能覆盖论文主要内容的通用技术词条(参照相应的技术术语 标准)。按词条的外延层次排列(外延大的排在前面)。摘要与关键词应在同一页。
% modifier: 黄俊杰(huangjj27, 349373001dc@gmail.com)
% update date: 2017-04-15
%%

\cabstract{
本研究旨在解决航空航天等领域流场特性分析中传统计算流体力学(CFD)方法计算成本高、效率低的问题,提出了一种创新的基于本征正交分解(POD)降阶模型的流场重构方法。从希尔伯特空间变分原理出发,建立基于 Snapshot POD 的自适应基函数生成框架,大幅降低流场维度。结合三次样条插值技术,构建完整的 POD 降阶模型(POD-ROM)流场重构框架,有效解决非采样工况下的模态耦合问题。进一步引入克里金(Kriging)预测方法,提升模型对复杂流场的预测能力。以 NACA0012 翼型为研究对象,运用专业 CFD 软件模拟生成多工况流场数据,对所提方法进行验证。数值实验结果表明,该方法在保证最大相对误差不超过 5\% 的前提下,计算效率较传统 CFD 方法提升了两个数量级。研究成果为无人机防除冰系统设计、气动性能优化等工程应用提供了有力的技术支撑,同时在计算效率和预测精度方面展现出显著优势,具有重要的工程应用价值。
}
% 中文关键词(每个关键词之间用“,”分开,最后一个关键词不打标点符号。)
\ckeywords{本征正交分解;降阶模型;流场重构;NACA0012 翼型;三次样条插值;计算流体力学;克里金模型}
\eabstract{
    This study aims to address the issues of high computational cost and low efficiency of traditional computational fluid dynamics (CFD) methods in the analysis of flow field characteristics in aerospace and other fields. An innovative flow field reconstruction method based on the proper orthogonal decomposition (POD) reduced-order model is proposed. Starting from the variational principle in Hilbert space, an adaptive basis function generation framework based on Snapshot POD is established, which greatly reduces the dimension of the flow field. By combining with the cubic spline interpolation technique, a complete flow field reconstruction framework of the POD reduced-order model (POD-ROM) is constructed, effectively solving the modal coupling problem under non-sampled conditions. Furthermore, the Kriging prediction method is introduced to improve the model's prediction ability for complex flow fields. Taking the NACA0012 airfoil as the research object, professional CFD software is used to simulate and generate flow field data under multiple working conditions to verify the proposed method. The results of numerical experiments show that this method can ensure that the maximum relative error does not exceed 5\%, and the computational efficiency is increased by two orders of magnitude compared with traditional CFD methods. The research results provide strong technical support for engineering applications such as UAV anti-icing system design and aerodynamic performance optimization. At the same time, it shows significant advantages in computational efficiency and prediction accuracy, and has important engineering application value.
}
% 英文文关键词(每个关键词之间用,分开, 最后一个关键词不打标点符号。)
\ekeywords{Proper Orthogonal Decomposition; Reduced-Order Modeling; Flow Field Reconstruction; NACA0012 Airfoil; Cubic Spline Interpolation; Computational Fluid Dynamics; Kriging Model}

