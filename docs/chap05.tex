\chapter{研究结论与展望}
\section{研究成果总结}
本研究围绕 NACA0012 翼型跨工况流场重构问题,构建了基于本征正交分解(POD)降阶模型的高效预测框架,通过理论推导、方法创新与数值验证,形成了系统性研究成果。首先,基于希尔伯特空间变分原理,建立了 Snapshot POD 自适应基函数生成框架。通过对高保真 CFD 模拟生成的多工况流场 “快照” 矩阵进行奇异值分解(SVD),提取累计能量占比达 95\% 以上的主导模态,成功将流场维度从原始高维空间降低 3 个数量级,显著提升了流场数据的压缩效率。所构建的低维特征空间能够准确捕捉翼型表面压力分布及流场压力分布的主要流动特征,前两阶表面压力模态和前九阶流场压力模态分别承载了对应物理场的关键信息。
其次,针对传统 POD 方法在非采样工况下的外推局限性,提出了融合三次样条插值的降阶模型(POD-ROM)。通过建立攻角、马赫数等参数与 POD 基系数的连续映射关系,构造了单变量及双变量三次样条插值模型,有效解决了模态耦合引起的预测偏差问题。数值实验表明,该方法在 NACA0012 翼型跨工况流场重构中表现优异,表面压力与流场压力的重构结果与 CFD 基准解高度吻合,最大相对误差控制在 5\% 以内,皮尔逊相关系数均超过 0.98,验证了方法的高精度特性。
此外,研究引入克里金(Kriging)预测模型与 POD 相结合,构建了多变量流场重构框架。通过将高维设计变量空间映射至低维 POD 系数空间,利用克里金模型的非线性插值能力,实现了对 19 个几何参数扰动下翼型流场的有效预测,表面压力预测的决定系数($R^2$)达到 0.956,为复杂参数空间下的气动分析提供了新的解决方案。


\section{研究的创新性}
本研究在流场重构方法体系及应用技术上实现了以下三方面创新:

(1)跨尺度建模框架的构建突破传统 POD 仅依赖模态截断的降维模式,提出基于 Snapshot POD 与三次样条插值的协同建模方法。通过希尔伯特空间内的变分优化,建立了自适应基函数生成机制,使 POD 基能够根据流场特征动态调整,相较经典 POD 方法,在保持 95\% 能量占比的前提下,将模态数量减少 40\%-60\%,显著提升了模型的泛化能力。

(2)参数空间的连续映射方法针对参数化流场重构中的 “维度灾难” 问题,首次将三次样条插值技术引入 POD 基系数的外推过程。通过构造满足$C^2$连续性的分段多项式函数,建立了扰动参数与模态权重之间的光滑映射,有效解决了非采样工况下的插值振荡问题。相较于线性插值,该方法使表面压力重构的均方根误差(RMSE)降低 30\%-40\%,为跨工况流场预测提供了更精确的数学工具。​

(3)多方法融合的预测体系创新性地将克里金模型与 POD 相结合,构建了适用于高维参数空间的流场重构方法。通过 POD 降维降低数据复杂度,利用克里金模型捕捉参数间的非线性耦合效应,实现了对翼型几何参数、来流条件等多变量扰动的高效处理。该融合方法在保持计算效率提升两个数量级的同时,将流场压力预测的平均绝对误差(MAE)控制在 2.5\% 以内,为多学科耦合优化提供了可靠的技术支撑。

\section{研究的局限性及改进措施}
尽管本研究取得了阶段性成果,但其理论框架与应用范围仍存在以下待改进之处:

(1)模型泛化能力的局限性当前研究聚焦于 NACA0012 对称翼型,其流场特征相对规则,而实际工程中复杂曲面翼型(如超临界翼型、结冰变形翼型)的流动分离、激波干扰等现象可能导致 POD 基的正交性退化,影响重构精度。此外,研究中采用的 k-ω 湍流模型在高雷诺数或强逆压梯度工况下的适用性有限,可能引入基准数据误差。

(2)参数空间的完备性不足研究仅考虑攻角、马赫数等主要气动参数,未涉及雷诺数、表面粗糙度、来流湍流度等因素,且未分析多参数强耦合对模态结构的影响。例如,结冰引起的翼型几何畸变会改变流场边界条件,现有模型难以直接应用。

(3)误差控制的精细化需求当前误差分析仅基于全局统计指标(如 RMSE、MAE),缺乏对局部流动特征(如分离涡位置、激波强度)的定量评估。此外,模态截断误差与插值误差的耦合机制尚不明确,尚未建立自适应的模态数选择策略。

针对上述问题,未来研究可从以下方向展开:

(1)扩展模型适用范围开展不同翼型(如 S809 风力机翼型、SC (2) 0712 高升力翼型)的流场重构研究,分析曲率、厚度分布对 POD 基特性的影响;引入动态 POD(DPOD)或扩展 POD(EPOD)技术,提升对非定常分离流、颤振流场的捕捉能力。​

(2)构建多物理场耦合模型将雷诺数、几何参数等纳入参数空间,采用拉丁超立方抽样扩展训练样本;结合计算几何技术,实现翼型参数化变形与流场重构的联动建模,为气动 - 结构耦合优化提供支撑。​

(3)优化误差控制与模型自适应开发基于局部流动特征的误差评估指标,如分离区压力梯度误差、涡核位置偏差;引入机器学习算法(如贝叶斯优化)动态调整模态数与插值节点,构建自适应性 POD-ROM 框架。
